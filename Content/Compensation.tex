\chapterimage{chapter_head_compensation.jpg} % Chapter heading image

\chapter{Greenhouse Gas Compensation}\label{chap:compensation}

The first principle for environmental protection from greenhouse gases should always be the reduction and avoidance of emission of those.  Nevertheless, some emissions can't be avoided without cancelling the whole project. At this point, compensation comes into play. 
Let's have a look at the different strategies of compensation!

\section{Carbon Emission Trading -- Reducing Allowed Emissions for Companies} 

In the 2009 UN Climate Change Conference in Copenhagen an at the most 2$^\circ$C (3.6$^\circ$F) rise in average temperature was set. \cite{copenhagen} To achieve this aim, calculated emission limits have to be met. Carbon emission trading was introduced by the Kyoto Protocol 1997 and is based on this strategy. Allowances are needed to release greenhouse gases into the atmosphere. A total amount of emissions, the cap, is set and allowances are distributed among companies and installations. The incentive is to reduce emissions to save money. If a company does not need all the bought allowances, it can sell them to other companies. This simple principle allows a market-based control of carbon emissions.
\cite{bmub_Emission_Trading}
 %(http://www.bmub.bund.de/en/topics/climate-energy/emissions-trading/general-information/ 10.5.17)


But how to compensate own emissions with this system? The idea is to buy and cancel allowances, making them unavailable for companies. This will lead to reduced emissions and companies are forced to invest into low carbon solutions. 
\cite{climakind}
%(http://climakind.com/c/2-What-is-Climakind.aspx 10.5.17)


Alright, let's take action at the global carbon market! But wait, there is none... There are 20 different emission trading systems existing, multinational, national and regional ones, making it quite difficult to follow the path. On their website, the International Emissions Trading Association (IETA) offers a good overview and detailed information about the different systems (see \url{http://www.ieta.org/The-Worlds-Carbon-Markets}).
Advantages of Emission Trading Systems are the straight-forwarded method and cheap prices as supporting climate protection projects have to be costly realized and monitored.


Different non-profit organizations have set themselves the target to destroy carbon allowances. Among them are Climakind, Sandbag and CO2compensation.org. 
All of them buy allowances from the European Union Emission Trade System (EU ETS), mainly because it was the first multinational ETS being established, leading to a nowadays highly reliable and robust system. Furthermore, the Carbon emissions in the EU ETS were reduced by 3.1\% in 2008, global emissions rose 1.9\% in the same time period. 
\cite{climakind_2} %(http://climakind.com/c/2-About.aspx 16.05.17) 

However, this does not indicate that other emission trading systems are not reliable, they have less experience and still have to prove that they meet the requirements set from these organizations. 

Despite the advantages of emission trading systems, Sandbag highlights weak points and has decided to temporarily close the Destroy Carbon Project. Reasons named are a low carbon price which is not forcing companies to invest into low carbon solutions, and over 3 billion tons of spare allowances in the EU ETS meaning that companies have still enough allowances to consume. After Sandbag, this system has led to even increased emissions in some cases. \cite{sandbag}


In summary, using emission trading systems for compensation of greenhouse gases follows a simple strategy and can be easily applied with cheap prices. Nevertheless, the 100 \% effect should not be taken as guaranteed and further information about different emission trading systems, the cap and spare allowances have to be gathered for each individual case. 


Interested in alternatives? No offence, but multiple values are important for you? Here comes the second option:

\section{Support Projects for Reduction of GHG Emissions}	

Besides emission trading, the Kyoto Protocol introduced Clean Development Mechanism (CDM) and Joint Implementation (JI) as market-based mechanisms that can be also used for compensation \cite{unfccc} %(http://unfccc.int/kyoto_protocol/mechanisms/items/1673.php 16.05.17). 
By implementation of an emission-reducing project in another country certified emission reduction credits can be earned and traded. The reduced emissions are then subtracted from the allowed ones, allowing countries with high emission of greenhouse gases to compensate. With CDM, the focus is laid on projects in developing countries. A Win-Win situation is created: investing countries have a flexible and cost-efficient way to meet the Kyoto commitments, host countries profit by investment and technology transfer.
The variety of organizations and companies offering climate protection projects is huge. Below, four options are introduced.

\begin{itemize}
	\item 
	\textbf{Climate Partner} is a company with multiple services for climate protection.  They invest money into certified projects with aims like protecting forests and establishing regenerative energies. For them, the benefit for people is as important as climate protection. Just look at their clean stove project in Peru: poor people suffer from enormous smoke generation while cooking over naked flame. By installing new and efficient stoves, both pollution and emission of greenhouse gases are reduced, less firewood is needed. \\
	Interested? Have a look at their website and choose your favorite project: \\ \url{http://www.climatepartner.com/en} 

	\item 
	\textbf{Plant for the planet} is an organization initiated and mainly operated by children worldwide. Their aim is to plant 1,000 billion trees around the world to absorb and store 10 billion tons of CO$_2$. They believe that trees serve as time-buffer to slow down climate change, giving us more time to effectively reduce our emission levels by usage of new technologies and adjusting life styles of industrial countries. \\
	Check out the different planting locations at: \\
	\url{https://www.plant-for-the-planet.org/en/home} 

	\item 
	\textbf{The Wilderness International foundation} has set themselves the target to preserve wilderness from industrial and agricultural damage. They buy land in West Canada to save the largest continuous area of temperate rainforest in the world. This region is endangered by timber and mining industries. GHG compensation is not their preferential aim, nevertheless protecting forest means saving trees which can compensate GHG. \\
	For more details, see: \url{http://wilderness-international.org/home}
	
	\item
	\textbf{CO$_2$ Neutral Website} is a company that compensates the emissions caused by your website, monitoring both servers and clients visiting it. It holds its own climate projects and more than 2500 companies worldwide are using their services.
	Have a look on their website: \url{https://www.co2neutralwebsite.com/}
\end{itemize}

Convinced? As already mentioned the options for compensation are numerous.

\begin{suggest}{Look for local projects and initiatives} 
	Maybe the park in your city needs a refresh and donations to plant new trees are welcome. Planting your own tree as symbolic action would also please us much.	
\end{suggest}


You see, the possibilities to compensate GHG are versatile and usually easy to perform, it's just some clicks away (\#noexcuse). Let's act responsible and sustainable together!


