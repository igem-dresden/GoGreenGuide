\chapterimage{chapter_head_intro.jpg} % Chapter heading image

\chapter*{Introduction}
\addcontentsline{toc}{chapter}{Introduction}\markboth{Introduction}{}%TODO check whether the markboth command should be added
\section*{How iGEM Goes Green Came Into Beeing}
\addcontentsline{toc}{section}{How iGEM Goes Green Came Into Beeing}\markboth{}{How iGEM Goes Green Came Into Beeing}
For more than 10 years now, over 300 teams from all over the world have been attending Boston to participate in the internationally most renowned student competition for synthetic biology: As part of the international Genetically Engineered Machine competition (iGEM) they find solutions to divers contemporary problems by using genetic components and biological systems.

Besides the lab work, a successful team also has to consider the aspects of  biological safety, public relations, documentation and finances of their project. However, when we set out to join iGEM with our own research project, we quickly realized being responsible for the safety, resources and outreach of our project wasn't covering a facet of the competition we feel strongly about: We decided to take one step further by taking responsibility for the environmental impact of our participation in iGEM and especially the related trip to Boston. We want to investigate the ecological footprint of our work as a team and find ways to reduce it. 

With the aim to encourage other teams to join us and get involved in sustainability themselves we decided to share our ideas and started the “iGEM goes green” initiative.

\section*{About this Guide}
\addcontentsline{toc}{section}{About this Guide}\markboth{}{About this Guide}
While this guide is written by an iGEM team and primarily with other iGEM teams in mind, it still strives to be of value to any other research group and project team interested in suggestions about a more environmentally friendly working routine as well.

However, for the purpose of giving other iGEM teams a general overview of what awaits them, \cref{chap:takingpart} is an iGEM-specific chapter focusing on how to participate in iGEM goes green. Any iGEM team not feeling motivated to read an entire chapter on the matter right away is invited to skip forward to \cref{checklist} for our \stress{GoGreenChecklist} for now.

\Cref{chap:lab} deals with possibilities for environmentally-conscious lab work. Due to the high security regulations in biological laboratories it isn’t possible to simply turn off devices and systems (e.g. ventilation) to save energy. Despite this, we still believe that lab work can be shaped into a more sustainable practice through good planning and the conscientious usage of resources. For now the chapter only includes basic information we gathered from different sources. Once our lab gets busy enough for it to yield reasonable results we will track our lab consumption of materials and energy for a set period of time to collect the data we need to calculate our ecological footprint and to identify possibilities to organise processes more economically. We will add more suggestions and some insights here once this is done.

As every other research project, iGEM is more then just lab work: regional meetups are an important and fixed part of every iGEM year and the highlight is the concluding conference in Boston in November, where all teams are presenting their results. \Cref{chap:meetup} contains notes concerning the sustainable organisation of meetings and conferences. 

Yet even if the conference was organized in the most sustainable way, there isn't much we can do to reduce the impact of the transatlantic flights from Germany to Boston and back. Therefore, \cref{chap:compensation} dedicates itself to possibilities to compensate CO$_2$ emission.

Everybody motivated to pursue the topic of sustainability further will find ideas going beyond the mere context of scientific research in chapter \cref{chap:more}. Last but not least \cref{chap:calculation} tries to provide at least basic knowledge on how to estimate a teams CO$_2$ footprint.
%TODO: Find nice concluding sentence. :)
