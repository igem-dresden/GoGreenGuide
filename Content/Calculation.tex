\chapterimage{chapter_head_calc.jpg} % Chapter heading image

\chapter{CO$_\text{2}$ Calculation}\label{chap:calculation}

The calculation of emission of CO$_2$ and other greenhouse gases is essential to understand the impact our actions have on the environment. Not only the obvious causes (e.g. intercontinental flights), but also the small ones (e.g. electricity) sum up to a final number. \\
Wait, we don't want to scare you! We are aware that the last sentence might cause readers to immediately stop and go over to the next chapter.  We solemnly swear to provide you with a reasonable compromise between simplicity and accuracy to get a good and easy estimation in the end. 


An important point to make things easy and feasible is setting boundaries about what should be included in calculations. The whole bunch of production processes of (subsub-) products might be left out, otherwise this calculation will take forever. \\
For iGEM teams the points labwork, meet-ups and (in our case the largest emission producer) flights should be taken into account. If you feel that an important point is missing, don't hesitate to make a suggestion at github. 

\section{Labwork}

We plan to measure the power consumption of gadgets and machines as well as the amount of consumables used. As our team has started working in the lab just a few weeks ago, we haven't had the opportunity for measurements yet. Therefore, this chapter is still under construction, we will inform you if an updated version of the Go Green Guide is available online.


Our plan is to provide you with an Exel sheet for easy estimation: just fill in consumed goods (pipette tips, tubes, ect.) and the duration different devices were used. Stay tuned, the detailed instructions on what to measure and count will follow soon.\\
By now, we are also cooperating with Wilderness International, a foundation located in our hometown Dresden, who plan to create an online tool for labwork-based emissions of greenhouse gases. The release is planned for October 2017, so  have a look on their website from time to time: 
\url{http://wilderness-international.org/}

\begin{suggest}{Wanna help?}
	The data we will obtain with our measurements can be seen as a general estimation.
	However, the deviation between different devices and different producers might be huge. Consequently own measurements are advantageous for a precise calculation.\\
	We ask you to track your lab consumption of materials and energy. This will also help us to provide average numbers and gain a better estimation for teams all over the world. \\
	Interested? Just contact us and we will support you and your team!
	
\end{suggest}

\section{Meet-ups}

The emissions caused due to meetings and other events strongly rely on duration, number of particpants and journey. \\
"Plant for the Planet" provides a useful online tool with the aim to have carbon neutral events. Therefore, the first step is the calculation of carbon emission, and that is exactly what we are looking for:

\begin{suggest} {}
	Go to \url{https://www.plant-for-the-planet.org/en/support/carbon-neutral-event)} and check how much carbon emissions are caused by your event. Have the following information ready:
	\begin{itemize}
		\item area of event location
		\item event duration
		\item number of participants
		\item transport (average distance, percentage of means of transport)
	\end{itemize}
\end{suggest}


\section{Flights and Travel}
Flights cause high emissions of greenhouse gases and therefore should be avoided whenever possible. Is there an alternative to get to the Giant Jamboree in Boston other then by plane (for those who don't live in North America)? By ship? *Joke* 


You see, flights usually can't be avoided. Nevertheless, nearly all means of transport (except for cycling and walking) cause emissions and in the end, all have to be taken into account. 
"Atmosfair", a German organization for climate protection, focuses on travel and provides a reliable online tool for travel emission calculations.


Below, we want to give a short list why we think the Atmosfair emission calculator is a trustful tool and why we highly recommend to apply it, especially for flights. 

\begin{itemize}
	\item A sufficient amount of independent data from scientific research projects was used to obtain numbers for nearly all combinations of factors. For example, if the user doesn't know the plane type, good estimations can be applied. Furthermore, factors with high impact on emissions are considered in detail, whereas factors with low impact are estimated.
	
	\item When burning kerosine, different pollutants having different effects are released (e.g. NO$_X$, CO$_2$ and particles). The resulting warming effect is calculated by transforming all effects into CO$_2$. Therefore the "\textbf{R}adiative \textbf{F}orcing \textbf{I}ndex" (RFI) is used. It can only be applied in high altitudes above 9km and is not used for climbing up and landing phases. The calculator strictly separates between those phases.
	
	\item The methodology is validated by Germany's Federal Environmental Agency, promising highly reliable calculations.
\end{itemize}

The detailed description of methods and estimations used in this emission calculator can also be found on their website as pdf file (Documentation_Calculator_EN_2008.pfd). \cite{flight_calc}

\begin{suggest} {}
	Go to \url{https://www.atmosfair.de/en/home} and check out your travel emissions. If you want to use a different tool, please check where they got the data from, how it works in detail and which estimations are made. Don't use it before evaluating its trustworthiness!
\end{suggest}

% hidden power guzzler, Green lab part