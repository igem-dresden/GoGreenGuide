\chapterimage{chapter_head_lab.jpg} % Chapter heading image

\chapter{Green Lab}\label{chap:lab}

With a high consumption of resources such as energy, water, and chemicals, laboratories are among the university institutions with the greatest impact on the environment. Depending on their size, laboratory buildings consume 3-4 times as much energy as an office building. Thereby, the largest share is accounted for ventilation and cooling systems (\raisebox{-0.9ex}{\~{ }} 60\%), followed by the lab equipment (freezer+ other devices) (\raisebox{-0.9ex}{\~{ }} 25\%) and most of the rest is lightning. \cite{GreenHarvard}

The total energy consumption is of course dependent on the size of the lab, the research topic, the equipment and its operating times. For calculating how much energy and resulting from that how much GHG your lab produces read \cref{chap:calculation} about Greenhouse Gas calculation. 
With some simple “lab hacks” you can avoid waste production, save energy and water and thereby reduce GHG emissions of your lab. 
If you are not convinced to change your behavior for the environment: sustainable laboratory work saves not only energy and resources, but also money.

\begin{suggest}{Find out how much GHG and energy your lab consumes!}
	Check out \cref{chap:calculation} on Greenhouse Gas calculation!
\end{suggest}

\section{Lab equipment}

\subsection{Fume hood}
One of the most energy intensive devices in laboratories are chemical fume hoods. The air changing system consists of supply fans that bring air in the fume hood and exhaust fans that pull the air out of the building.  A fume hood that runs 24 hours a day, 365 days a year consumes 3.5-times more energy than an average house! 

\begin{suggest}{Close the sash of your fume hood each time you stop working there and remind your lab mates to do the same!}
	Closing the window in front of the fume hood, known as “sash” can reduce the exchanged air from 600 m$^{3}$ to 200 m$^{3}$ per hour. Thereby, a lot of energy and money can be saved.
\end{suggest}


Harvard University started the “Shut the Sash” program in 2005 and even published their own study with data and behaviour tips to reduce the costs and increase the energy efficiency of fume hoods.\cite{Harvard_sash}
The University of Nottingham made a similar 'shut the sash" campaign and could save 5\%-25\% of energy.
\cite{Nottingham_sash}


\subsection{Freezer}
Where do you store your biological samples? In the freezer of course. 
Ultra low temperature freezers are commonly used to store biological samples over a longer time period. Thereby, they can cost more than \EUR{1000} in ‘plug load’ electricity (i.e. not including their impact on room air cooling systems). This is just one $-80^\circ \text{C}$ freezer!
How many freezers do you have in your laboratory building? 
Best saving measure is to reduce the number of freezer running at the same time or make them run more efficiently. \\

\textbf{But how to achieve that?  }

\begin{suggest}{Become a sample minimalist}
		Start decluttering your freezer and organize it in a more accessible way. Make sure you know exactly what is inside and dispose any samples that are no longer needed. Do regular inventory checks before it starts getting messy again. 
\end{suggest}

\begin{suggest}{Share with a neighboring lab}
	After decluttering you have now a freezer that is only partly filled? Ask your neighbor labs if you can share freezing space with them. Supports socializing as well.
\end{suggest}

\begin{suggest}{Defrost and clean your freezers regularly}
	Opening your freezer gives you an arctic feeling and makes you afraid of polar bear attacks? Time for defrosting!\\ 
	When there is an ice layer covering the coils, the compressor must run longer to maintain its temperature = higher energy consumption.
\end{suggest}

\begin{suggest}{Raise freezer temperatures}
	Raising the temperature from -80 to -70 degrees saves up to 30\% of the energy it uses and research shows that many samples can be stored at -70 degree as well. Moreover, the less cooling effort will increase the lifetime of your freezer. 
\end{suggest}

\begin{suggest}{Find the right place for your freezer}
	Always remember that cooling the interior also means heating the outside, and this heat needs to go somewhere. So if you store your freezers in a dedicated cold room, the air-conditioning system is adjusted to remove the heated air directly. The room temperature will not be very affected.
	But if your freezer is in the hallway, you're just heating the whole environment and your air-conditioning has to work against this 24/7. So think carefully and find the right place for your freezers and refrigerators.
\end{suggest}	

\begin{suggest} {Store your samples at room temperature}
	Yes, this is possible! Especially DNA and RNA can be stored long-term in a dried state without quality loss. This also applies for bacteria in short-term means and might be soon addressed for other biological samples. Storing the dried samples in well-plates can shrink the container size up to 90\%. \cite{RT_storage} Stanford University estimated how much savings would be possible if room temperature storage was implemented. For their 9 million to 13 million samples that are potential candiates, room temperature storage would save 2.4 million kWh (1100 tonnes of CO$_2$ emissions) per year! \cite{stanford}
\end{suggest}	

\subsection{Autoclave} 
A single run of an autoclave can consume up to 900 liters of water, but of course sterilization methods are important for save working. Still by improving your autoclaving skills, you can reduce energy and water consumption of the device significantly.
 
Consider the following rules when autoclaving:

\begin{suggest}{Do not run the autoclave only half filled but do not overfill the chamber either – Steam Flow is Key!}
	Whenever you need to run the autoclave, ask your lab mates to combine load and leave a note at the autoclave to remind other people to ask you before autoclaving, too. 

\end{suggest} 

\begin{suggest}{Turn the unit off and keep the door shut when not in use }
	Even when not in use, an autoclave uses between 3 to 6 liters per minute for its cooling system only.\\ 
	You would not leave the door of your refrigerator open, would you? 
	
\end{suggest} 

\begin{suggest}{Use energy saving control features }
	Check the instructions of your device to find out if there is a more ecofriendly mode for your use applicable.
	
\end{suggest}	

\begin{suggest}{Sort your stuff! }
	Reduce autoclaving needs by sorting your working material. Only required items are autoclaved and other items are run through the dishwasher.
\end{suggest}	

%TODO: Add point about several runs in a row without letting it cool down all the way.

\subsection{Electrical devices}
	There are so many different electrical devices in a lab you only need on certain days or just for some hours a day. 
	What about centrifuges, PCR cyclers, plate readers, heating plates ...? 
	
	
\begin{suggest}{Remember to turn them off every time you finish your task!} 
	You can also add "Turn me off" stickers to the devices to remind your lab mates. 
	Also turn off environmental rooms or incubators when not in use.\\ 
	The Kirschner's Lab at Harvard Medical School did a case study and revealed that turning off benchtop devices over night can save more than 50 \% of energy consumption. \cite{Kirschner_1} \cite{Kirschner_2}
\end{suggest}	

\begin{suggest}{Share your devices with others}
	If your device is not booked 24/7, why not offer it to other scientists? This might cause higher energy consumption for you in the first place, but prevents a potential new electricity guzzler in another lab. Among others, Open IRIS is a free platform to enable ressource sharing. Keep it in mind for your next experiment! \cite{IRIS}
\end{suggest}

\begin{suggest}{Look out for labels}
	We all know different labels, e.g. for organic food or energy efficient devices. There are initiatives to develop sustainability labels for lab products. If you're in the US, make sure to buy freezers that have the energy STAR label. 
	Further labels are the EGNATON-CERT label and the ACT label: \\
	EGNATON-CERT is a work in progress, but there will be a sustainable label for e.g. autoclaves, freezers, washing machines or incubators. Launch of the first certified products is planned in 2018. \\
	The ACT label was developed in the US to label sustainable lab products as consumables, chemicals and reagents and equipment. It is also still in progress (launch 09/2017). Find out more about the different labels on their websites: \\
	\url{http://act.mygreenlab.org/0} \\
	\url{http://www.egnaton.com/en/Home.aspx}
\end{suggest}
	

\begin{suggest}{Give your old devices a second life}
	Donate old machines, especially to educational institutions such as high schools, instead of throwing them out. Giving away unneeded devices reduces problems like hazardous waste in landfills. For example, an old incubator could be refurbished for use in high schools with cell cultures rather than be broken down in a landfill.	
\end{suggest}	
	
\section{Lab consumables}
Plastic accumulates in ever larger quantities in the oceans, but because of its advantages it is difficult to replace. 
Plastic materials are also widely used in every day lab work. Due to sterility, products like gloves and pipettes are often used only once. 
So it is more important than ever to think about waste reduction, recycling and environmental friendly behavior concerning lab consumables.

\subsection{Tips from A-Z. Let's get started!} 

\begin{abc}
	\suggestion{Autoclavable glassware}
	All sorts of lab equipment are also available as glass ware instead of disposable plastic products. Do you really need one-use plastic pipettes or wouldn't glass pipettes serve the same purpose? As already mentioned in the previous section, autoclaving is an energy and water consuming process. 
	Autoclaving reusable glassware is still more efficient than autoclaving contaminated plastic waste. 

	\suggestion{Biodegradable materials}
	The theory behind bioplastics is simple: 
	Making plastic starting from kinder chemicals, it would break down faster and easier when we compost it. 
	Check out if things like biodegradable gloves are practicable for you!

	\suggestion{Consolidate purchase}
	Reduce packaging waste and transport ways by combining your orders with other labs, by planning ahead or by ordering from local suppliers. Additionally, if sterility is not an issue, try to buy items that are packed in bulk.

	\suggestion{Dry orders}
		Order your oligos dry. It saves shipping weight, package material and is more stable. 

	\suggestion{Eliminate hazardous chemicals}
	Minimize the generation of hazardous wastes and find out if you could use eco-friendlier chemicals instead. 
	E.g the MIT Green Chemical Alternatives Wizard assists researchers to find a safer, more environmentally-friendly alternative. %TODO: add link

	\suggestion{Follow iGEM goes green }
	We'll keep you updated about revisions of the Guideline, new tips and ideas and more!
	
	Instagram: \url{https://www.instagram.com/igem_goes_green}\\
	Facebook:  \url{https://www.facebook.com/igem.goesgreen}

	\suggestion{Get creative!}
	Give your package material a second life and reuse card boxes and styrofoam containers. Upcycle them as storage boxes, underlay or hiding place for your cat.   

	\suggestion{Have a look if your supplier cares about sustainability}
	Do they have solutions regarding packaging waste, resource-efficient and sustainable products and what about their company? \\
	Many companies take care on environmental and social issues - have a look on their website, their "sustainability report" or just talk to their representatives. If you have the choice, select the company who cares most on sustainable issues.
	

	\suggestion{Inspire}
	Set a good example by making your lab more sustainable and motivate coworkers or other labs to join the movement. \#igemgoesgreen

	\suggestion{Journal}
	Purchase journal abonnements online and do not print articles to reduce paper waste. 

	\suggestion{Keep it simple!} 
	Start with simple changes in your daily working routine to make your lab work greener - then you can deal with bigger problems. 

	\suggestion{Label your stuff!}
	Label everything you do clearly, so you and others can be sure what it is and if it is still needed. 
	Also label your bins clearly to support waste separation and recycling. 

	\suggestion{Minimize}
	Order only the amount that you need. Do not buy chemicals in large amounts just because it is cheaper when you will never use it up. 

	\suggestion{No excuses! }
	Start making your lab more sustainable today. There are so many possibilities how to start right away. 

	\suggestion{Offer...}
	... unwanted equipment or unused materials to other labs or student groups. 

	\suggestion{Pipette tips}
	Refill your old tip boxes by purchasing your tips in bulk. It makes great as initiation ritual for new lab members or as pastime during lab meetings. Alternatively, old pipette boxes in the lab make great containers for storage. 

	\suggestion{Quality}
	Make sure that all your lab consumables are of a high quality. %TODO: how is that good for the environment?

	\suggestion{Reduce experiments}
	Reduce the scale of experiments and protocols to the minimum size necessary to achieve research objectives. 
	This can be achieved by good planning and saves chemicals, tubes and other working material.

	\suggestion{Size}
 	Use appropriate sized vessels to store your samples. You will save plastic, freezer space and money.

	\suggestion{Tubes}
	Buy your micro reaction vessels and tubes from recycled plastic. Get them in bags and refill boxes and racks yourself.

	\suggestion{Up-to-date inventory}
	Keep an up-to-date list of your lab consumables to avoid duplicate orders.

	\suggestion{Values}
	Try to instill green values in coworkers and support green suppliers. 
	
	\suggestion{Waste manager}
	Choose a person responsible for waste reduction and informing other people about recycling ways. 


	\suggestion{XYZ...}
	Have a closer look around your lab. You will probably find more ways to reduce, recycle and refill. 

\end{abc}

\subsection{Further important information}
Some topics haven't been covered in the alphabet yet, and they also need more space for explanation. What happens to materials after use? And have you heard about Green Chemistry before?

\begin{suggest}{Green Chemistry}
	Green Chemistry, also knows as sustainable chemistry, aims to design chemical processes and products in such a way that no hazardous substances are generated. This means that remediation (treating waste streams and cleaning releases), is NOT included. Instead, less hazardous chemicals should be used.\\
	For more information, click on the links below:
	\begin{itemize}
		\item Find the main 12 principles here: \\ \url{https://www.epa.gov/greenchemistry/basics-green-chemistry#twelve}
		\item If you're planning a synthesis, you can use the "Toolbox" to find the most sustainable synthesis pathway:  \url{https://www.chem21.eu/project/metrics-toolkit/}
		\item Solvents can be hazardous for your health and the environment. Use less toxic alternatives - this solvent guide can help you to make the right choice: \url{https://www.chem21.eu/project/chem21-solvent-selection-guide/}
		\item If you would like to learn more about green chemistry, have a look on the Chem21 online learning platform: \url{http://learning.chem21.eu/#}
	\end{itemize}
\end{suggest}

\begin{suggest}{Green analytics and Green chromatography}
	"Green chemistry" was the basis to transfer it to analytics and chromatography and to frame the principles of green analytics:
	\begin{itemize}
		\item Replace and reduce toxic solvents, reagents, preservatives, additives for pH adjustement,  avoid derivatization of your samples, minimize waste  and reduce your ressource consumption during your experiments.
		\item Miniaturization is key! Minimize your samples size and number of samples! This could be achieved by using microextraction techniques like SPME (Solid Phase Microextraction) or LPME (Liquid Phase Microextraction).  Other examples are Nano-LC or capillary electrophoresis (e.g.for DNA-/RNA-electrophoresis).
		\item To minimize your number of samples, try to get as much information out of your sample as possible - 2D-chromatography is a suitable tool for all scientists working with complex samples (environmental analytics, etc.).
		\item Integration and automatization of processes also helps in information gathering, minimizes the variability between individual experiments and ensures safer procedures.
	\end{itemize}
	Find more information on the principles of green analytical chemistry in the publication of A. Gałuszka \textit{et al.} (2013). \cite{green_analytics}
\end{suggest}

\begin{suggest} {Waste separation and recycling}
	As stated before, hazardous chemicals and substances should be avoided at all costs. However, this is not possible in all cases and waste separation comes into play. Try to divert waste from the landfill by learning which objects can be recycled or composted and which cannot. A safety chart in a laboratory may also be effective for determining where waste should go, what should be recycled, and what should be considered hazardous waste. \\
	Also consider that plastics could be collected separately, recycled and used for eco-responsible durable goods. Does your supplier have a "Take back"-program ? Here's a recycling program for gloves and garments: \url{https://www.kcprofessional.com/brands/kimtech/rightcycle}
\end{suggest}	
	

\section{Workspace}

 Much has been said about how to go about working greener in the lab, but what about the office? Research is also a lot of reading, writing and computer work. Check out your possibilities for a more sustainable desk space. 
 
\subsection{Paper and printing products}
	
	Nearly every office relies on large quantities of paper. On the one hand paper is made from wood, a truly renewable and sustainable resource. However, on the other hand the current demand for paper consumes already 40\% of the annual industrial wood harvest. So we exhaust the resource faster than it can regrow.  \cite{WWF} This will not work forever we should start overthinking our paper production and consumption practices now. 

\begin{suggest}{Alternatives?}
	Reduce paper waste by avoiding paper in the first place. 
	Be more aware whenever you use paper and think about whether you really need a paper version of whatever you're up to.
\end{suggest}

\begin{suggest}{Store manuals, policies and other documents digitally}
		 Consider the environment before printing emails, papers or instructions. 
\end{suggest}

\begin{suggest}{Use both sides of paper}
	You thought about it and you are absolutely sure that you need a printed version of your document? Then at least make sure that your printer prints double-sided. Use it as default setting for your office printer.
\end{suggest}

\begin{suggest}{Use recycled paper}
	Buying a writing pad with 80 sheets of recycled paper instead of a conventional one 
	saves the amount of CO$_2$ that 353 web queries produce.
	More advantages of recycling paper compared to paper made from wood:  
	\begin{itemize}
		\item  The production requires 2-6 times less water.
		\item  The total energy consumption is 3-4 times lower. 
		\item  The resource wood is spared. \cite{Paper}
	\end{itemize}  
\end{suggest}
	
\begin{suggest}{Reuse unwanted paper}
	E.g shred it and use it as package material for shipments. 
\end{suggest} 

\subsection{Lights}
	Turn the lights off whenever you leave a room. 
	
\begin{suggest}{Natural lightning}
	Try to make use of the cheapest light source and shut the light in your office off as long as there is enough daylight for working. Use the night for sleeping and work during the day.
\end{suggest} 

\begin{suggest}{LED}
		 Save energy and money by switching to LED lights. They are not very expensive anymore and their life span is so much longer than other lights.
\end{suggest}

\subsection{Computers}
	Your computer is running the whole day - even when you are in the lab? 
	Think about when you use your computer the most and switch it off in between.

\begin{suggest}{Computer Sharing}
	Car sharing is a common trend, so how about computer sharing? 
	Estimate the time you spend in front of the monitor and think about if you really need your own computer. Maybe you can manage to share one with your lab mate. 
\end{suggest}

\begin{suggest}{Give it a nap}
	Make sure your computer turns to energy saving mode when you are not using it for a short time. 
\end{suggest}

Thanks for reading this chapter. You see, it is not that difficult to make a difference! If you feel motivated to also change some of your personal habits now, check out \cref{chap:more}.

Main ideas from this chapter were taken from the Harvard Green Labs Program \cite{GreenHarvard} and the Green Lab Program of the University of California \cite{Greenlab}. \\

Do you want to implement many tips but don`t know whether your institution will support you in this case?
Do they have an environmental or sustainability management staff position ? If yes, get in contact ! Stabilize your project and ideas - in the end it could become a permanent element of your university organisation. 
Here's an example how to integrate and consolidate a "green lab program": \\ \url{http://www.colorado.edu/ecenter/greenlabs}
