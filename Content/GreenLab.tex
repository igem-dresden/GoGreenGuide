\chapterimage{chapter_head_lab.jpg} % Chapter heading image

\chapter{Green Lab}\label{chap:lab}

With a high consumption of resources such as energy, water, and chemicals, laboratories are among the university institutions with the greatest impact on the environment. Depending on their size, laboratory buildings consume 3-4 times as much energy as an office building. Thereby, the largest share is accounted for ventilation and cooling systems (~60\%), followed by the lab equipment (freezer+ other devices) (~25\%) and most of the rest is lightning. 

The total energy consumption is of course dependent on the size of the lab, the research topic, the equipment and its operating times. For calculating how much energy and resulting from that how much $CO_2$ your lab produces read the chapter about $CO_{2}$ calculation. 
With some simple “lab hacks” you can avoid waste production, save energy and water and thereby reduce $CO_{2}$ emissions of your lab. 
If you are not convinced to change your behavior for the environment: sustainable laboratory work saves not only energy and resources, but also money.

\section{Lab equipment}

\subsection{Fume hood}
One of the most energy intensive devices in laboratories are chemical fume hoods. The air changing system consists of supply fans that bring air in the fume hood and exhaust fans that pull the air out of the building.  A fume hood that runs 24 hours a day, 365 days a year consumes 3.5-times more energy than an average house! (http://sustainability.ucr.edu/certification/greenlab.html) Closing the window in front of the fume hood, known as “sash” can reduce the exchanged air from 600 m3 to 200 m3 per hour. Thereby, a lot of energy and money can be saved.
Close the sash of your fume hood each time you stop working there and remind your lab mates to do the same! 


Harvard University started the “Shut the Sash” program in 2005 and even published their own study with data and behavior tips to reduce the costs and increase the energy efficiency of fume hoods:      
https://green.harvard.edu/sites/green.harvard.edu/files/FumeHoodWhitePaper.pdf

\subsection{Freezer }
Where do you store your biological samples? In the freezer of course. 
Ultra low temperature freezers are commonly used to store biological samples over a longer time period. Thereby, they can cost more than \EUR{1000} in ‘plug load’ electricity (i.e. not including their impact on room air cooling systems). This is just one $-80^\circ \text{C}$ freezer!
How many freezers do you have in your laboratory building? 
Best saving measure is to reduce the number of freezer running at the same time or make them run more efficiently. 

But how to achieve that?  

\begin{enumerate}
	\item \textbf{Become a sample minimalist}\\
	Start decluttering your freezer and organize it in a more accessible way. Make sure you know exactly what is inside and dispose any samples that are no longer needed. Do regular inventory checks before you have a mess again. 
	
	\item \textbf{Share with a neighboring lab}\\
	After decluttering you have now a freezer that is only partly filled? Ask your neighbor labs if you can share freezing space with them. Supports socializing as well.
	
	\item \textbf{Defrost and clean your freezers regularly}\\
	Opening your freezer gives you an arctic feeling and makes you afraid of ice bear attacks? Time for defrosting!\\ 
	When there is an ice layer covering the coils, the compressor must run longer to maintain its temperature = higher energy consumption
	
	\item \textbf{Raise freezer temperatures}\\
	Raising the temperature from -80 to -70 degrees saves up to 30\% of the energy it uses and research shows that many samples can be stored at -70 degree as well. Moreover, the less cooling effort will increase the lifetime of your freezer. 
	
\end{enumerate}

\subsection{Autoclave} 
Sterilization methods are important for save working Improve handling your autoclave. A single run of autoclave can consume up to 900 liters of water. By improving your autoclaving skills, you can reduce energy and water consumption of the device significantly. 
Consider the following rules when autoclaving: 

\section{Lab consumables}
