\chapterimage{chapter_head_lab.jpg} % Chapter heading image

\chapter{Green Lab}\label{chap:lab}

With a high consumption of resources such as energy, water, and chemicals, laboratories are among the university institutions with the greatest impact on the environment. Depending on their size, laboratory buildings consume 3-4 times as much energy as an office building. Thereby, the largest share is accounted for ventilation and cooling systems (\raisebox{-0.9ex}{\~{ }} 60\%), followed by the lab equipment (freezer+ other devices) (\raisebox{-0.9ex}{\~{ }} 25\%) and most of the rest is lightning. \cite{GreenHarvard}

The total energy consumption is of course dependent on the size of the lab, the research topic, the equipment and its operating times. For calculating how much energy and resulting from that how much GHG your lab produces read the chapter about Greenhouse Gas calculation. %TODO querverweis
With some simple “lab hacks” you can avoid waste production, save energy and water and thereby reduce GHG emissions of your lab. 
If you are not convinced to change your behavior for the environment: sustainable laboratory work saves not only energy and resources, but also money.

\begin{suggest}{Find out how much GHG and energy your lab consumes!}
	Check out chapter 6 on Greenhouse Gas calculation! %TODO Querverweis
\end{suggest}

\section{Lab equipment}

\subsection{Fume hood}
One of the most energy intensive devices in laboratories are chemical fume hoods. The air changing system consists of supply fans that bring air in the fume hood and exhaust fans that pull the air out of the building.  A fume hood that runs 24 hours a day, 365 days a year consumes 3.5-times more energy than an average house! Closing the window in front of the fume hood, known as “sash” can reduce the exchanged air from 600 m$^{3}$ to 200 m$^{3}$ per hour. Thereby, a lot of energy and money can be saved.

\begin{suggest}{Close the sash of your fume hood each time you stop working there and remind your lab mates to do the same!}
\end{suggest}


Harvard University started the “Shut the Sash” program in 2005 and even published their own study with data and behavior tips to reduce the costs and increase the energy efficiency of fume hoods:   \\ \url{https://green.harvard.edu/sites/green.harvard.edu/files/FumeHoodWhitePaper.pdf} %TODO Link

\subsection{Freezer}
Where do you store your biological samples? In the freezer of course. 
Ultra low temperature freezers are commonly used to store biological samples over a longer time period. Thereby, they can cost more than \EUR{1000} in ‘plug load’ electricity (i.e. not including their impact on room air cooling systems). This is just one $-80^\circ \text{C}$ freezer!
How many freezers do you have in your laboratory building? 
Best saving measure is to reduce the number of freezer running at the same time or make them run more efficiently. \\

\textbf{But how to achieve that?  }

\begin{suggest}{Become a sample minimalist}
		Start decluttering your freezer and organize it in a more accessible way. Make sure you know exactly what is inside and dispose any samples that are no longer needed. Do regular inventory checks before it starts getting messy again. 
\end{suggest}

\begin{suggest}{Share with a neighboring lab}
	After decluttering you have now a freezer that is only partly filled? Ask your neighbor labs if you can share freezing space with them. Supports socializing as well.
\end{suggest}

\begin{suggest}{Defrost and clean your freezers regularly}
	Opening your freezer gives you an arctic feeling and makes you afraid of ice bear attacks? Time for defrosting!\\ 
	When there is an ice layer covering the coils, the compressor must run longer to maintain its temperature = higher energy consumption.
\end{suggest}

\begin{suggest}{Raise freezer temperatures}
	Raising the temperature from -80 to -70 degrees saves up to 30\% of the energy it uses and research shows that many samples can be stored at -70 degree as well. Moreover, the less cooling effort will increase the lifetime of your freezer. 
\end{suggest}

\subsection{Autoclave} 
Sterilization methods are important for save working Improve handling your autoclave. A single run of autoclave can consume up to 900 liters of water. By improving your autoclaving skills, you can reduce energy and water consumption of the device significantly.
 
Consider the following rules when autoclaving:

\begin{suggest}{Do not run the autoclave only half filled but do not overfill the chamber either – Steam Flow is Key!}
	Whenever you need to run the autoclave, ask your lab mates to combine load and leave a note at the autoclave to remind other people to ask you before autoclaving, too. 

\end{suggest} 

\begin{suggest}{Turn the unit off and keep the door shut when not in use }
	Even when not in use, an autoclave uses between 3 to 6 liters per minute for its cooling system only.\\ 
	You would not leave the door of your refrigerator open, would you? 
	
\end{suggest} 

\begin{suggest}{Use energy saving control features }
	Check the instructions of your device to find out if there is a more ecofriendly mode for your use applicable.
	
\end{suggest}	

\begin{suggest}{Sort your stuff! }
	Reduce autoclaving needs by sorting your working material. Only required items are autoclaved and other items are run through the dishwasher.
\end{suggest}	

\subsection{Electrical devices}
	There are so many different electrical devices in a lab you only need on certain days or just for some hours a day. 
	What about centrifuges, PCR cyclers, plate readers, heating plates ...? 
	
	
\begin{suggest}{Remember to turn them off every time you finish your task!} 
	You can also add "Turn me off" sticker to the devices to remind your lab mates. 
	Also turn off environmental rooms or incubators when not in use.
\end{suggest}	

	
\section{Lab consumables}
Plastic accumulates in ever larger quantities in the oceans, but because of its advantages it is difficult to replace. 
Plastic materials are also widely used in every day lab work. Due to sterility, products like gloves and pipettes are often used only once. 
So it is more important than ever to think about waste reduction, recycling and environmental friendly behavior concerning lab consumables.
Tips from A-Z. Let's get started! 

\begin{abc}
	\suggestion{Autoclavable glassware}
	All sorts of lab equipment are also available as glass ware instead of disposable plastic products. Do you really need one-use plastic pipettes or wouldn't glass pipettes serve the same purpose? As already mentioned in the previous section, autoclaving is an energy and water consuming process. 
	Autoclaving reusable glassware is still more efficient than autoclaving contaminated plastic waste. 

	\suggestion{Biodegradable materials}
	The theory behind bioplastics is simple: 
	Making plastic starting from kinder chemicals, it would break down faster and easier when we compost it. 
	Check out if things like biodegradable gloves are practicable for you!

	\suggestion{Consolidate purchase}
	Reduce packaging waste and transport ways by combining your orders with other labs or by planning ahead. Additionally, if sterility is not an issue, try to buy items that are packed in bulk.

	\suggestion{Dry orders}
		Order your oligos dry. It saves shipping weight, package material and is more stable. 

	\suggestion{Eliminate hazardous chemicals}
	Minimize the generation of hazardous wastes and find out if you could use eco-friendlier chemicals instead. 
	E.g the MIT Green Chemical Alternatives Wizard assists researchers to find a safer, more environmentally-friendly alternative. 

	\suggestion{Follow iGEM goes green }
	Follow the iGEM goes green account on Facebook or Instagram to find more tips on how to reduce waste.
	
	Instagram: \url{https://www.instagram.com/igem_goes_green/}\\
	Facebook:  \url{https://www.facebook.com/igem.goesgreen/}

	\suggestion{Get creative!}
	Give your package material a second life and reuse card boxes and styrofoam containers. Upcycle them as storage boxes, underlay or hiding place for your cat.   

	\suggestion{Halogenated waste}
	Segregate halogenated from non-halogenated wastes.

	\suggestion{Inspire}
	Set a good example by making your lab more sustainable and motivate coworkers or other labs to join the movement. \#gogreen

	\suggestion{Journal}
	Purchase journal abonnements online and do not print articles to reduce paper waste. 

	\suggestion{Keep it simple!} 
	Start with simple changes in your daily working routine to make your lab work greener - then you can deal with bigger problems. 

	\suggestion{Label your stuff!}
	Label everything you do clearly, so you and others can be sure what it is and if it is still needed. 
	Also label your bins clearly to support waste separation and recycling. 

	\suggestion{Minimize}
	Order only the amount that you need. Do not buy chemicals in large amounts just because it is cheaper when you will never use it up. 

	\suggestion{No excuses! }
	Start making your lab more sustainable today. There are so many possibilities how to start right away. 

	\suggestion{Offer...}
	... unwanted equipment or unused materials to other labs or student groups. 

	\suggestion{Pipette tips}
	Refill your old tip boxes by purchasing your tips in bulk. It makes great as initiation ritual for new lab members or as pastime during lab meetings. Alternatively, old pipette boxes in the lab make great containers for storage. 

	\suggestion{Quality}
	Make sure that all your lab consumables are of a high quality.

	\suggestion{Reduce experiments}
	Reduce the scale of experiments and protocols to the minimum size necessary to achieve research objectives. 
	This saves chemicals, tubes and other working material.

	\suggestion{Size}
 	Use appropriate sized vessels to store your samples. You will save plastic, freezer space and money.

	\suggestion{Tubes}
	Buy your micro reaction vessels and tubes from recycled plastic. Get them in bags and refill boxes and racks yourself.

	\suggestion{Up-to-date inventory}
	Keep an up-to-date list of your lab consumables to avoid duplicate orders.

	\suggestion{Values}
	Try to instill green values in coworkers and support green suppliers. 
	
	\suggestion{Waste manager}
	Choose a person responsible for waste reduction and informing other people about recycling ways. 


	\suggestion{XYZ...}
	Have a closer look around your lab. You will probably find more ways to reduce, recycle and refill. 

\end{abc}

\section{Workspace}

 Much has been said about how to go about working greener in the lab, but what about the office? Research is also a lot of reading, writing and computer work. Check out your possibilities for a more sustainable desk space. 
 
\subsection{Paper and printing products}
	
	Nearly every office relies on large quantities of paper. On the one hand Paper is made from wood, a truly renewable and sustainable resource. But the current demand for paper consumes already 40\% of the annual industrial wood harvest. So we exhaust the resource faster than it can regrow.  \cite{WWF} This will not work forever we should start overthinking our paper production and consumption practices now. 

\begin{suggest}{Alternatives?}
	First of all: reduce paper waste by avoiding the use of it. 
	Be more aware whenever you use paper and think about if you really need paper version of whatever you're up to. 
\end{suggest}

\begin{suggest}{Store manuals, policies and other documents online}
		 Consider the environment before printing emails, papers or instructions. 
\end{suggest}

\begin{suggest}{Use both sides of paper}
	You thought about it and you are absolutely sure that you need a printed version of your document? Then at least make sure that your printer prints double-sided. Use it as default setting for your office printer.
\end{suggest}

\begin{suggest}{Use recycling paper}
	Buying a writing pad with 80 sheets of recycling paper instead of a conventional one 
	saves the amount of CO$_2$ that 353 web queries produce.
	More advantages of recycling paper compared to paper made from wood:  
	\begin{itemize}
		\item  The production requires 2-6 times less water.
		\item  The total energy consumption is 3-4 times lower. 
		\item  The resource wood is spared. \cite{Paper}
	\end{itemize}  
\end{suggest}
	
\begin{suggest}{Reuse unwanted paper}
	E.g shred it and use it as package material for shipments. 
\end{suggest} 

\subsection{Lights}
	Turn the lights off whenever you leave a room. 
	
\begin{suggest}{Natural lightning}
	Try to make use of the cheapest light source and shut the light in your office off as long as there is enough daylight for working. Use the night for sleeping and work during the day. 
\end{suggest} 

\begin{suggest}{LED}
		 Save energy and money by switching to LED lights. They are not very expensive anymore and their life time is so much longer than other lights.
\end{suggest}

\subsection{Computers}
	Your computer is running the whole day - even when you are in the lab? 
	Think about when you use your computer the most and switch it off in between.

\begin{suggest}{Computer Sharing}
	Car sharing is a common trend, so how about computer sharing? 
	Estimate the time you spend in front of the monitor and think about if you really need your own computer. Maybe you can manage to share one with your lab mate. 
\end{suggest}

\begin{suggest}{Give it a nap}
	Make sure your computer turns to energy saving mode when you are not using it for a short time. 
\end{suggest}


Thanks for reading this chapter. Did you see how easy it is to work more sustainable? 
If you feel now more responsible and you want to include sustainability to your personal life as well, check out Chapter 5!  %TODO Querverweis

Main ideas from this chapter were taken from the Harvard Green Labs Program \cite{GreenHarvard} and the Green Lab Program of the University of California \cite{Greenlab}