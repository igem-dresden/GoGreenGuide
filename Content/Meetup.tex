\chapterimage{chapter_head_meetup.jpg} % Chapter heading image

\chapter{Green Meetups}\label{chap:meetup}
This chapter concerns especially conferences - in the iGEM context teams from certain areas get together for local meetups to create contacts and collaborations. However, even if you do not plan to host a conference of any kind yourself any time soon, this chapter is worth reading. Many aspects will be relevant for simple team meetings or events just as well. Furthermore, if you are attending a conference, reading this chapter will give you a good understanding on how to help shaping it green.

\section{Transportation}
Depending on the distance and mode of travelling, transportation easily becomes a very big issue for sustainable meetings. Try reducing the traffic induced environmental stress of you meeting as much as possible and think about compensating for unavoidable emissions or suggest to do so to the participants of you meeting. There are easy calculation tools for emissions produced by transportation and lots of providers for compensation, check out \cref{chap:calculation} and \cref{chap:compensation}.

Keep in mind the following:

\begin{suggest}{Make sure, public transport is a good option}
	\vspace{-2\topsep}
	\begin{itemize}
		\item Choose a location that is easily reachable by public transport.
		\item Organising a shuttle service or shared cars form the train/bus station if the final location is hard to reach without a car.
		\item Schedule beginning and end of your event according to the operating times of relevant public transport
		\item Give ``attractive'' options for using public transport. Check with the provider if there are any discounts. The tickets could include a bike option in the city etc.
	\end{itemize}
\end{suggest}

\begin{suggest}{Provide support for planning the journey}
	\vspace{-2\topsep}
	\begin{itemize}
		\item Inform the participants about environmental friendly travel options, this includes giving them a detail description on how to reach the location via public transport. The best choice can vary in each situation. Guiding principle: bike > a full bus > a full car > train > an empty car with just one person > plane
		\item Encourage them to share cars. If possible, provide a platform (mailing list, facebook group etc) for organizing shared rides.
	\end{itemize}	
\end{suggest}

\begin{suggest}{Considering possible alternatives like telephone or video conferences}
	Obviously this is not exactly an option for iGEM meetups, but meetings without a connected social aspect often don't call for a meeting in person.
\end{suggest}

\section{Accommodation and Energy}
2.1 Goals
*	Compensating the Carbon Emissions
*	Measures for efficient energy use
2.2 Measures
*	Ask the accommodations wether they are using „green“ energy sources, choose the (cheapest) hotel with the best energy conditions
*	Not heating the conference rooms over 20$^\circ$C, not cooling them under 6 degrees less then the outside temperature
*	If you have the choice use energy efficient devices

\section{Acquisitions of services on products}
3.1 Goals 
*	Taking to account the environmental stress caused by every acquisition 
3.2 Measures
*	Avoiding paper waste, print on both sites, minimal number of handouts, and take back and recycle or reuse flyers etc.
*	Try to use 100% recycled paper
*	If tendering for services state your environmental goals
\section{Food and Catering}
4.1 Goals
*	Significant percentage of organic products
*	Supporting seasonal groceries 
*	Supporting fair trade products
4.2 Measures
*	Favouring organic and fair trade (especially coffee and chocolate) products
*	Using seasonal products with short transportations routes
*	Considing that especially animal products cause exceptionally high green house gas emissions, so try to reduce the amount of meat, dairy products and eggs involved.
*	A vegetarian or vegan option should always be provided. If you want to go all the way: Make the whole event vegetarian.
*	Avoiding fish from endangered resources  (look at the MSC quality control)
*	Providing tap water in carafes

\section{Waste management} 
5.1 Goals
*	Avoidance and reduction of waste
*	Usage of environmental friendly packaging like reusable bottles
*	Stopping the common „paper flood“ happening regularly at conferences 
5.2 Measures
*	Set up a place for waste separations 
*	Use environmental friendly packaging
*	Use reusable plates, cutlery and glasses

\section{Water usage}
6.1 Goals
*	Protect the resource water
6.2 Measures
*	Spare use of water
*	Providing a guideline how to safe water in the bathrooms

\section{Communication}
7.1 Goals
*	Assuring the success of providing a sustainable „green“ meeting
7.2 measures
*	Naming a person responsible for the “green” aspects oft the event, that everybody can reach with questions
*	Extensive public relations: The goal of having a „green“ meeting and the approaches to do so should be made public early on. This works as a stimulus to reach the stated goals. It's also a good advertisement for your meeting and for the green movement in general
*	All of the participants should be informed about the green aspects of the meeting beforehand 

\section{Social aspects}
8.1 Goals
*	Taking in account the needs of people with disabilities
*	Taking in account the pricipals of “Gender Mainstreaming” meaning making sure there are no strucal disadvantages for either gender  
8.2 Measures
*	Taking care of the accessibility for wheelchairs including the toilets
*	 People with special needs (blindness, impaired hearing ect.) should be provided with help to enable their participation
*	Using gender-neutral language for verbal and written statements




