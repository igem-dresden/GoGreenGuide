\chapterimage{chapter_head_meetup.jpg} % Chapter heading image

\chapter{Green Meetups}\label{chap:meetup}
This chapter concerns especially conferences and is mainly based on the guide for sustainable organization of events by the German Federal Environmental Agency \cite{meeting}. In the iGEM context teams from certain areas get together for local meetups to create contacts and collaborations. However, even if you do not plan to host a conference of any kind yourself any time soon, this chapter is worth reading. Many aspects will be relevant for simple team meetings or events just as well. Furthermore, if you are attending a conference, reading this chapter will give you a good understanding on how to help shaping it green.

\section{Transportation}
Depending on the distance and mode of travelling, transportation easily becomes a very big issue for sustainable meetings. Try reducing the traffic induced environmental stress of you meeting as much as possible and think about compensating for unavoidable emissions or suggest to do so to the participants of you meeting. There are easy calculation tools for emissions produced by transportation and lots of providers for compensation, check out \cref{chap:calculation} and \cref{chap:compensation}.
	
Keep in mind the following:

\begin{suggest}{Make sure, public transport is a good option.}
	\vspace{-2\topsep}
	\begin{itemize}
		\item Choose a location that is easily reachable by public transport.
		\item Organising a shuttle service or shared cars form the train/bus station if the final location is hard to reach without a car.
		\item Schedule beginning and end of your event according to the operating times of relevant public transport.
		\item Give ``attractive'' options for using public transport. Check with the provider if there are any discounts. The tickets could include a bike option in the city etc.
	\end{itemize}
\end{suggest}

\begin{suggest}{Provide support for planning the journey.}
	\vspace{-2\topsep}
	\begin{itemize}
		\item Inform the participants about environmental friendly travel options, this includes giving them a detail description on how to reach the location via public transport. The best choice can vary in each situation. Guiding principle: bike > a full bus > a full car > train > an empty car with just one person > plane
		\item Encourage them to share cars. If possible, provide a platform (mailing list, facebook group, etc.) for organizing shared rides.
	\end{itemize}	
\end{suggest}

\begin{suggest}{Considering possible alternatives like telephone or video conferences.}
	Obviously this is not exactly an option for iGEM meetups, but meetings without a connected social aspect often don't call for a meeting in person.
\end{suggest}

\section{Accommodation and Energy}

Saving energy is something most people have thought about before, if only for financial reasons. Of course it's a good idea in the context of protecting the environment as well:

\begin{suggest}{Reduce energy consumption.}
	\vspace{-2\topsep}
	\begin{itemize}
		\item Don't heat the conference rooms over 20$^\circ$C, don't cool them down further than 6 degrees below outside temperature.
		\item If you have the choice use energy efficient devices.
	\end{itemize}	
\end{suggest}

However, one thing you probably don't think about much when you plug in a device or turn on the light is where exactly the energy comes from. Green energy makes quite a difference \cite{energy}.

\begin{suggest}{Think about energy sources.}
	Ask the accommodations whether they are using ``green'' energy sources, choose the (cheapest) hotel and conference building with the best energy conditions.
\end{suggest}

\section{Food and Catering}

\begin{suggest}{Favour organic, fair trade and local products in season.}
	The positive impact of seasonal products is obvious (short transport routes). However, though you will probably agree that organic and fair trade are great in general, you might wonder how both help reducing emission. To answer that: Fairtrade International developed a Climate Standard and actively works against climate change \cite{fairtrade} and organic farming, amongst other benefits, doesn't use the pesticides and fertilizers common in conventional farming which have a far larger carbon footprint \cite{organic}.
\end{suggest}

\begin{suggest}{Provide tab water in carafes.}
	Aside from the environmental benefits it safes you money and the hassle of transporting all those heavy water bottles.
\end{suggest}

\begin{suggest}{Reduce the amount of animal products on the menu.}
	Global livestock causes 14.5 percent of all anthropogenic GHG emissions \cite{livestock}. That's good reason to reduce the amount of meat, dairy products and eggs involved in your meals. Lamb, beef and cheese have the biggest impact here \cite{animalGHG}. Yes, that's right: Cheese is worse than chicken, so going veg isn't enough (though of course it helps).
	\begin{itemize}
		\item Always provide vegetarian and vegan options.
		\item Cut down on animal products. It's up to you how far you want to go here - however, eating veg or even vegan for a weekend shouldn't be much of a problem for anyone. ;)
		\item Pay special attention to the source of products like coffee, chocolate and fish.
	\end{itemize}
\end{suggest}

\section{Material and Services}
\begin{suggest}{Procure and use materials prudently.}
	\vspace{-2\topsep}
	\begin{itemize}
	\item Avoid paper waste: Print on both sites, minimize the number of handouts, and take back and recycle or reuse flyers etc.
	\item Try to use 100\% recycled paper.
	\end{itemize}
\end{suggest}

\begin{suggest}{Pay attention to the sustainability of providers.}
	When inviting offers for services or products always state your environmental goals.
\end{suggest}

\section{Waste management} 
Conferences and get-togethers with big enough numbers of people almost always end up with paper floods and mountains of disposable cutlery, plates and cups. We have talked about handouts and flyers before. 

\begin{suggest}{Reduce waste where you can...}
	\vspace{-2\topsep}
	\begin{itemize}
	\item Use environmental friendly packaging (avoid plastic, prefer reusable containers, buy in bulk).
	\item Use reusable plates, cutlery and glasses.
	\end{itemize}
\end{suggest}

\begin{suggest}{...and separate where you can't.}
	Set up places for waste separation and make them easy to use by clearly stating what goes where and making sure full containers get emptied quickly.
\end{suggest}

\section{Communication}
Welcome to the most important section of this chapter. You might have read the preceding sections with a bit of unease or doubt, especially when it comes to checking energy sources or having all vegan meals. Understandably. If you haven't, maybe because none of this is new to you and you are all excited to realize all this, keep in mind this probably won't be the case for all participants of your meeting. Therefore, no matter how far you decide to go with having a green meeting, make sure the communication is working out. Everyone involved needs to know about your goal to have a sustainable ``green'' meeting, otherwise it will be hard to make it a success.

\begin{suggest}{Inform participants and public early on.}
	The goal of having a ``green'' meeting and the approaches to do so should be made public early on. While this works as a stimulus to reach the stated goals it's also a good advertisement for your meeting and for the green movement in general. Especially the participants should be informed about the green aspects of the meeting beforehand.
\end{suggest}

\begin{suggest}{Involve the participants.}
	You won't be successful without their support and cooperation. Provide them with ways to get involved easily. We mentioned providing ways to organize car sharing before. You can think of more! Provide short guidelines on how to safe water in the rest rooms for example.
\end{suggest}

\begin{suggest}{Name a person responsible for the ``green'' aspects oft the event.}
	That way someone will have the overview about what's going on. Everyone involved in organizing the event as well as participants should know whom to contact with questions.
\end{suggest}