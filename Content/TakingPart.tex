\chapterimage{chapter_head_takepart.jpg} % Chapter heading image

\chapter{Taking Part in iGEM Goes Green}\label{chap:takingpart}

\section{Join the initiative!}
Officially becoming a part of iGEM goes green is fairly simple. All you've got to do is:
\begin{enumerate}
	\item Dedicate a subpage of your team's wiki to your efforts towards a sustainable participation in the competition. 
	\item Send us the link.
	\item Get the go green logo in return and place it on your wiki.
	\item Do whatever your team can afford to shape your teams participation in iGEM in a green way.
\end{enumerate}
We plan on having a map showing all teams involved on our wiki and using it to link to their go green subpages. As the environmental impact and the options for reducing and compensating greenhouse gases vary a lot between teams of different locations, sizes and projects, having specific requirements or comparing teams doesn't make much sense. Therefore it's entirely up to your team, how much you actually do and what steps towards a green iGEM competition seem feasible to you. You decide whether your team is part of iGEM goes green.


\section{Set a sign - Plant a tree!}\label{sec:trees}

When we first thought about compensating the environmental impact of our flights to Boston - before we started iGEM goes green - our first idea was, of course, to plant trees. While we realized we would go beyond the mere compensation of our flights we also figured out that the flights alone would take several hundreds of trees to make up for. Still the idea remained and we liked the thought of planting a tree somewhere in Dresden or its vicinity at least as a symbolic act. Right now we are assessing possibilities. Our vision is to have an ``iGEM forest'' with a tree for every iGEM team participating in iGEM goes green. We'll keep you updated whether we'll be able to organize this. If so we'll ask every go green team to collect the money for one tree. If not we encourage you to search for a spot somewhere close to your university for planting a tree for your team yourself.

\section{GoGreenChecklist}\label{checklist}
To give you a good overview of what awaits you, we wrote this checklist. You can put it up in your teams head quarter or use it to show your efforts on your wiki's go green subpage.
\bigskip
{\renewcommand{\labelitemi}{\tickleaf}
\begin{checklistbox}
{\sffamily\textcolor{ocre}{Required to officially become a part of iGEM goes green:}}
\begin{itemize}
	\setlength{\itemsep}{-0.5\parsep}
	\item[\tickleafticked] Read this guide.
	\item Create a wiki subpage for your efforts towards a green competition.%TODO more there:
	\item Send us the link.
\end{itemize}
\end{checklistbox}
\bigskip
All set? Great to have you on board! No wiki yet? Doesn't matter. There is no reason not to start with the actual work already and send us the link later on. Now you can start with the actual actions toward green iGEM and fill your wiki's subpage with what you are doing. Below are possible steps and suggestions. Depending on your teams situation you might not have the possibilities to realize all of this, but don't worry, every little bit counts. Little by little a little becomes a lot!
\bigskip
\begin{checklistbox}
{\sffamily\textcolor{ocre}{Actions towards a green iGEM competition:}}
\begin{itemize}
	\setlength{\itemsep}{-0.5\parsep}
	\item Make your lab environmentally friendly (suggestions in \cref{chap:lab}).
	\item Organize your meetings and meetups environmentally friendly (see \cref{chap:meetup}).
	\item Track your labs consumption of materials and energy\dots
	\begin{itemize}
		\item[] \dots to be able to estimate your teams GHG emission.
		\item[] \dots to share the data with us.
	\end{itemize}
	\item Calculate the GHG emission caused by your teams trip to Boston.
	\item Calculate your teams entire GHG footprint (don't worry, we tell you how to do it in \cref{chap:calculation})
	\item Compensate your teams flights, part of or even your entire emissions (check \cref{chap:compensation} for options).
	\item Help to spread the word by sharing and following iGEM goes green on social media and by posting your own updates.
	\item Involve the public even more by organizing events.
	\item Help improving this guide by correcting mistakes or even sending us additional ideas, suggestions or information. Of course you'll be mentioned as a collaborator.
	\item Collect the money to plant a tree. Please read up on our tree-planting plans in \cref{sec:trees}.
	\item your ideas: {\leavevmode\leaders\hrule  height -1.9pt depth 2pt \hfill\kern0pt\relax}
\end{itemize}
\end{checklistbox}
}
%TODO add more references to chapters etc.

\section{What's in it for you?}

Taking part in iGEM goes green is of course not about winning anything. It's about being a part of change, about setting a sign and shouting out that we do care about our environment. Of course you will gain more then just a good feeling: Sustainability is the future and you'll become quite an expert concerning sustainable bench science. You will also get the chance to collaborate in a field outside your actual project by helping us to develop this guide further and to gather enough data to learn something about the average environmental profile of an iGEM team. You'll get connected with other teams on a whole new level.

Still, most important is the fact that you will help to lessen the impact of scientific work on the environment. Let's rephrase the question: What's in it for the environment? And that's not really a question, is it?
