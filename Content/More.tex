\chapterimage{chapter_head_more.jpg} % Chapter heading image

\chapter{Green Life}\label{chap:more}
 
Nowadays there are so many lifestyles addressing different aspects of sustainable living. Minimalism, Veganism and Zero Waste are just some catch words here. 

In this chapter we collected a few suggestions for developing a more sustainable lifestyle to provide a starting point for everyone who has not yet thought about sustainability in everyday life but feels motivated now to dive into the topic.

\section{Eating Habits} 
 	What we buy and eat influences regional and global structures. Sustainable nutrition means considering the health aspects and the environmental, ecological and social impact of your food \cite{food}. The same facts we already mentioned in \cref{chap:meetup} are applicable here:
 	
\begin{suggest}{Reduce the amount of meat, eggs and dairy products on your menu}%TODO: Include "Zurück zum Sonntagsbraten" here?
 	You do not have to quit everything you enjoy from your meal schedule. But think about reducing your consumption of animal-based food and pay attention to the quality. Global livestock causes 14.5\% of all anthropogenic GHG emissions \cite{livestock}.
 	A plant-based mixed diet produces about 15\% less green house gases than an unbalanced meat-based diet. Moreover, the production requires less water and acreage \cite{food}. 
\end{suggest}
 	
\begin{suggest}{Buy regional and seasonal products}
  	Buying regional and seasonal avoids unnecessary food transport. 
  	Furthermore it supports regional agriculture and economy. 
\end{suggest}

\begin{suggest}{Favour organic and fair trade}
	Though you will probably agree that organic and fair trade are great in general, you might wonder how both help reducing emission. To answer that: Fairtrade International developed a Climate Standard and actively works against climate change \cite{fairtrade}, and organic farming, amongst other benefits, doesn't use the pesticides and fertilizers common in conventional farming which have a far larger carbon footprint \cite{organic}.
\end{suggest}

\begin{suggest}{Drink tab water (if that's safe in your country)}
	It safes you money, the hassle of transporting all those heavy water bottles and is good for the environment. Also get yourself a reusable bottle for whenever you are on the go. Having your own fancy bottle might even motivate you to drink enough during the day. :)
\end{suggest}

\section{Transport}
	Being mobile is crucial these days for most people, no question here. However, a car is not the only option to get around of course.
\begin{suggest}{Go by bike whenever possible...}
	Traffic studies show that riding a bike can save up to 138g CO$_{2}$  per km and that up to 30\% of the car trips in cities could be replaced by bike rides\cite{bike}. And it also has more advantages than just being climate-friendly: Riding a bike is flexible, pretty fast (for shorter trips within the city you might even beat public transport sometimes), healthy and makes a good start of the day. You'll feel much more energized and less stressed when you arrive at work or university after a morning bike ride instead of squeezing into the too full morning bus or going by car during rush hour.
\end{suggest}

\begin{suggest}{...and choose the best alternative when the bike isn't an option}
	The best choice can vary in each situation.	A general guiding principle is: bike > a full bus > a full car > train > a rather empty car > plane.\footnote{For Germany the ``Deutsche Bahn''  offers a calculation service: \url{http://www.deutschebahn.com/en/sustainability/environmental_pioneer/umweltschutz_interaktiv/11887484/mobilecheck.html}} Think about looking for shared rides or offering them yourself, whenever you want to go by car.
\end{suggest}

\section{Waste Reduction}
\begin{suggest}{Switch to reusable bags}
	Most plastic bags are used only once to carry home your purchases. But after that it takes decades for the plastic bag to disappear. But even then they don't truly disappear they only break down into tiny pieces you don't see anymore but which will end up in the environment and everywhere in the food chain \cite{plasticbags}. And we haven't even talked about the related GHG emission during production yet! So bring reusable bags when going shopping to avoid wasting plastic bags.
\end{suggest}

\begin{suggest}{Think twice before you buy a new product.}
	You might realize your old stuff is still working just fine. When you do buy something new, better spent a little more money and get something lasting instead of any of the cheap stuff that will only end up in landfills too soon. 
\end{suggest} 

\begin{suggest}{Avoid plastic wrapping.}
	Shops that allow you to buy all your groceries in your own reusable containers can be found in some places already. But even in common stores you will find options using less plastic wrapping than others. Also bring reusable bags for veggies and fruits and try to avoid those plastic bags provided in the fruit and vegetable department as much as possible.
\end{suggest}

\clearpage

\section{Further Reading} 

\begin{suggest}{The lazy persons guide for saving the world:} 
\url{http://www.un.org/sustainabledevelopment/takeaction/}
\end{suggest}

\begin{suggest}{Green blog published by The Green Guide Institute (TGGI):}
\url{http://blog.thegreenguide.com/}
\end{suggest}
