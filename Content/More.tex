\chapterimage{chapter_head_more.jpg} % Chapter heading image

\chapter{Green Life}\label{chap:more}
Now we have already thought so much about making your lab greener and more sustainable. 
In our times there a so many lifestyles adressing different aspects of sustainable living. 
Minimalizm, Veganism and Zero waste are just some key words. 
In this chapter we would like to give you a few some suggestions to develop a more sustainable lifestyle.


\section{Consider your eating habits} 
 	What we buy and eat influences regional and global structures. 
 	Sustainable nutrition means considering the health aspects, environmental, ecological and social impact of your food. \cite{food}
 	
\begin{suggest}{Reduce you meat consumption}
 	You do not have to quit everything you enjoy from your meal schedule.
 	But think about if you could reduce consuming animal-based food or at least pay attention to the quality. 
 	A plant-based mixed diet produces about 15\% less green house gases than an unbalanced meat-based diet. Moreover, the production requires less water and acreage. \cite food
\end{suggest}
 	
\begin{suggest}{Buy regional and seasonal products}
  	Buying regional and seasonal avoids unescessary food transports. 
  	Futhermore it supports regional agriculture and economy. 
\end{suggest}

\section{Transport}
	Living without mobility and transport is not imaginable anymore. 
\begin{suggest}{Ride your bike or take public transportation.}
	Traffic studies show that riding a bike can save up to 138 g $CO_{2}$  per km and up to 30\% of the car trips in cities could be replaced by bike rides.\cite{bike}
	Furthermore, riding a bike is fast, healthy and climate-friendly. 
\end{suggest}

\section{Reduce waste}
	Reduce your need to buy new products. If there is less waste, then there is less to recycle or reuse.  
	If you buy new things or food try to avoid plastic wrapped items. Plastic never goes away. 

\begin{suggest}{Switch to reusable bags}
	Most plastic bags are used only once to carry your purchases from the shop to your home. But after that it takes millions of years for the plastic bag to decompose.
	So bring reusable bags when going shopping to avoid wasting plastic bags. 
\end{suggest}


\section{Further reading} 

\begin{suggest}{The lazy persons guide for saving the world:} 
\url{http://www.un.org/sustainabledevelopment/takeaction/}
\end{suggest}

\begin{suggest}{Green blog published by The Green Guide Institute (TGGI):}
\url{http://blog.thegreenguide.com/}
\end{suggest}
